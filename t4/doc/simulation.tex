\section{Simulation Analysis}
\label{sec:simulation}
In addition to theoretical analysis, we computed a simulation of the audio amplifier using Ngspice, and measured the output voltage gain in the passband, the lower and upper 3dB cut off frequencies, and the input and output impedances. The respective results are shown in Tables \ref{tab:spice} and \ref{tab:imped}

%inserir grafico
\begin{figure}[H] \centering
\includegraphics[width=0.5\linewidth]{vo2f.eps}
\caption{Gain in function of frequency response}
\label{fig:sim1}
\end{figure}

In addition, Figure~\ref{fig:sim1} illustrates the gain as a function of the frequency in log10 scale.

\begin{table}[H]
 \centering
 \begin{tabular}{|l|r|}
 \hline
 {\bf Name} & {\bf Value} \\ \hline
\input{../sim/spice_results_tab}
 \end{tabular}
 \caption{Voltage gain, bandwidth and lower cutoff frequency.}
 \label{tab:spice}
 \end{table}
 
 \begin{table}[H]
 \centering
 \begin{tabular}{|l|r|}
 \hline
 {\bf Name} & {\bf Value} \\ \hline
\input{../sim/op_2_tab}
\input{../sim/op_in_tab}
 \end{tabular}
 \caption{Input and output impedances of the circuit.}
 \label{tab:imped}
 \end{table}
 
 In the Table.......... we can check that $V_{CE}>V_{BE} and $V_{EC}>V_{EB}$ by presenting the respective values. This allows us to conclude that the transistors operate in the forward active region (F.A.R.).

\begin{table}[H]
 \centering
 \begin{tabular}{|l|r|}
 \hline
 {\bf Name} & {\bf Value [V]} \\ \hline
@cb[i] & 0.000000e+00\\ \hline
@ce[i] & 0.000000e+00\\ \hline
@q1[ib] & 7.022567e-05\\ \hline
@q1[ic] & 1.404513e-02\\ \hline
@q1[ie] & -1.41154e-02\\ \hline
@q1[is] & 5.765392e-12\\ \hline
@rc[i] & 1.411536e-02\\ \hline
@re[i] & 1.411536e-02\\ \hline
@rf[i] & 7.022567e-05\\ \hline
@rs[i] & 0.000000e+00\\ \hline
v(1) & 0.000000e+00\\ \hline
v(2) & 0.000000e+00\\ \hline
base & 2.254108e+00\\ \hline
coll & 5.765392e+00\\ \hline
emit & 1.411536e+00\\ \hline
vcc & 1.000000e+01\\ \hline

 \end{tabular}
 \caption{Verification of the FAR.}
 \label{tab:forwardbias}
 \end{table}

\begin{table}[H]
 \centering
 \begin{tabular}{|l|r|}
 \hline
 {\bf Name} & {\bf Value} \\ \hline
\input{../sim/merito_tab}
 \end{tabular}
 \caption{Cost, quality and merit of the circuit.}
 \label{tab:merit_spice}
\end{table}

