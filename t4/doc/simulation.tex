\section{Simulation Analysis}
\label{sec:simulation}

In addition to theoretical analysis, we computed a simulation of the AC/DC converter using Ngspice, and plotted the output voltages of the envelope detector and voltage regulator circuits. The results are illustrated in Figure~\ref{fig:venvout}.

\vspace{-30mm}

\begin{figure}[H] \centering
\includegraphics[width=0.6\linewidth]{../sim/venvout.pdf}
\caption{Plots of the output voltages of the envelope detector (blue) and voltage regulator (red) circuits obtained by simulation.}
\label{fig:venvout}
\end{figure}

So as to observe the deviation of the output signal from $12V$, we also plotted the Ngspice reproduction of $v_o-12$, which we can see in Figure~\ref{fig:vdiff}.

\vspace{-20mm}

\begin{figure}[H] \centering
\includegraphics[width=0.6\linewidth]{../sim/vdiff.pdf}
\caption{Deviation of the output signal from $12V$ obtained by simulation.}
\label{fig:vdiff}
\end{figure}

Furthermore, we managed to determine the average of $v_o$ and its respective ripple. The average deviation from $12V$ and the ripple of the output voltage obtained by simulation are presented in Table~\ref{tab:results}, side by side to the corresponding values obtained previously with Octave.

\vspace{2mm}

\begin{center}
\begin{tabular}{ | c | c | c | }\hline
\input{../mat/simtab} 
\end{tabular}
\captionof{table}[]{Average deviation from $12V$ and ripple of the output voltage obtained theoretically and by simulation.}
\label{tab:results}
\end{center}

\vspace{2mm}

Comparing the theoretical results to the ones obtained by simulation, one notices that the average deviation from $12V$ is approximately $0$ for the first one and exactly $0$ for the second one. Similarly, the ripple calculated with Octave is very close to the one computed with Ngspice. With this in mind, and taking into account that the plots presented in both sections match almost precisely, it becomes evident that both methods are in agreement in terms of the obtained results.\\

Finally, to calculate the merit $M$ of the converter, we used the values obtained by simulation. They should be more reliable, since in the theoretical analysis we considered several approximations that may have affected the results. The merit $M$, as well as the values required to its calculation, are illustrated in Table~\ref{tab:merit}.

\vspace{2mm}

\begin{center}
\begin{tabular}{ | c | c | }\hline
\input{../mat/merit} 
\end{tabular}
\captionof{table}[]{Merit of the converter.}
\label{tab:merit}
\end{center}

