\section{Introduction}
\label{sec:introduction}

The objective of this laboratory assignment is to design an Audio Amplifier, which increases the input voltage, making it as efficient as possible in what concerns both the cost and functioning. For that purpose, we had at our disposal resistors, capacitors, and NPN and PNP transistors, whose costs are presented in Table~\ref{tab:costs}.

\begin{center}
\begin{tabular}{ | c | c | }\label{tab:costs}
Component & Cost \\
\hline
Resistors & $1$ MU/k$\Omega$ \\  
Capacitors & $1$ MU/$\mu$ F \\
Transistors & $0.1$ MU/transistor    
\end{tabular}
\captionof{table}[]{Cost of the components of the circuit.}
\end{center}

The following quantity, merit $M$, is defined with the purpose of evaluating the quality of the audio amplifier:

\begin{equation}
  M = \frac{voltageGain \times bandwidth}{cost \times lowerCutoffFreq}
\end{equation}

Thus, the higher the value of the function, the better its quality.\\

With this in mind, we built the circuit illustrated in Figure~\ref{fig:amplifier}, with a resistor with resistance $R = 369.982$ $k\Omega$, a capacitor with capacitance $C = 420$ $\mu F$ and $4+19$ regular diodes.

\begin{figure}[H] \centering
\includegraphics[width=1\linewidth]{amplifier.pdf}
\caption{Audio Amplifier.}
\label{fig:amplifier}
\end{figure}

In Section~\ref{sec:analysis}, we present a theoretical analysis of the audio amplifier, where we develop a model able to predict the gain, input and output impedances separately for both the gain and the output stages. Here, we also get the theoretical frequency response using the incremental circuit. In Section~\ref{sec:simulation}, we write an Ngspice script to simulate the Audio Amplifier and to measure the aforementioned quantities. Additionally, we compare the results from Section~\ref{sec:analysis} to the ones obtained by simulation. Furthermore, in Section~\ref{sec:conclusion}, we summarise the results of the simulations and their agreement with the theoretical predictions. Finally, we present the merit $M$ of the designed amplifier.

