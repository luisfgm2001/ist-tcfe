\section{Simulation Analysis}
\label{sec:simulation}

In this part of the assignment, we computed a simulation of the BandPass Filter in Ngspice in order to obtain a central frequency of 1000 Hz and a gain at this frequency of 40 dB.

As we proceeded previously, we started by computing the frequency response of the circuit. The plots of the gain for both analysis are presented in Figures~\ref{fig:gainngspice} and ~\ref{fig:gainoctave} side by side.

\vspace{-25mm}

\noindent
\begin{minipage}[c]{0.5\linewidth}

\begin{figure}[H] \centering
\includegraphics[width=1\linewidth]{../sim/vdbf.pdf}
\caption{Frequency Response - Gain (Ngspice)}
\label{fig:gainngspice}
\end{figure}

\end{minipage}
\begin{minipage}[c]{0.5\linewidth}

\vspace{16mm}

\begin{figure}[H] \centering
\includegraphics[width=1\linewidth]{gainoctave.eps}
\caption{Frequency Response - Gain (Octave).}
\label{fig:gainoctave}
\end{figure}

\end{minipage}

COMPAREMOS OS GRÁFICOS NAO ESQUECER

\vspace{10mm}

We also plotted the phase of the frequency response for both methods in Figures~\ref{fig:phasengspice} and ~\ref{fig:phaseoctave}:

\vspace{-25mm}

\noindent
\begin{minipage}[c]{0.5\linewidth}

\begin{figure}[H] \centering
\includegraphics[width=1\linewidth]{../sim/phasef.pdf}
\caption{Frequency Response - Phase (Ngspice)}
\label{fig:phasengspice}
\end{figure}

\end{minipage}
\begin{minipage}[c]{0.5\linewidth}

\vspace{16mm}

\begin{figure}[H] \centering
\includegraphics[width=1\linewidth]{phaseoctave.eps}
\caption{Frequency Response - Phase (Octave).}
\label{fig:phaseoctave}
\end{figure}

\end{minipage}

COMPAREMOS OS GRÁFICOS NAO ESQUECER OUTRA VEZ

\vspace{10mm}

Afterwards, we obtained the low cutoff frequency in order to calculate the central frequency.

We then calculated the output voltage gain, input and output impedances at the central frequency and the deviations of the gain and central frequency. The results of both theoretical and simulated methods are in Tables~\ref{tab:resultsngspice} and ~\ref{tab:resultsoctave}.

\vspace{5mm}

\noindent
\begin{minipage}[c]{0.5\linewidth}

\begin{table}[H]
 \centering
 \begin{tabular}{|l|r|}
 \hline
 {\bf Name} & {\bf Value} \\ \hline
%%%%%%\input{../sim/data_tab}
%%%%%%\input{../sim/Zout_tab}
 \end{tabular}
 \caption{Results of the simulation (Ngspice)}
 \label{tab:resultsngspice}
 \end{table}
 
\end{minipage}
\begin{minipage}[c]{0.5\linewidth}

\vspace{-5mm}
 
 \begin{table}[H]
 \centering
 \begin{tabular}{|l|r|}
 \hline
 {\bf Name} & {\bf Value} \\ \hline
%%%%%\input{../mat/totaltab}
 \end{tabular}
 \caption{Results of the theoretical analysis (Octave)}
 \label{tab:resultsoctave}
 \end{table}
 
\end{minipage}

\vspace{10mm}

Finally, to calculate the merit $M$ of the BandPass Filter, we used the values obtained by simulation. They should be more reliable, since in the theoretical analysis we considered several approximations that may have affected the results. The merit $M$, as well as the cost of the circuit, are illustrated in Table~\ref{tab:merit}. In what concerns the cost, we took into account the cost of the components of the OP-AMP.

\vspace{2mm}

\begin{center}
\begin{tabular}{ | c | c | }\hline
%%%\input{../sim/merit_tab} 
\end{tabular}
\captionof{table}[]{Cost and merit of the BandPass Filter.}
\label{tab:merit}
\end{center}
