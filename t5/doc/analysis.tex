\section{Theoretical Analysis}
\label{sec:analysis}

In order to analyse theoretically the circuit illustrated in Figure~\ref{fig:bandpass}, we considered that the OP-AMP is ideal and, therefore, its input impedance is infinite and its output impedance is zero. Keeping this in mind, we were able to derive the following equations, respectively, for the input and output impedances of the circuit:

\begin{equation}
Z_{in}=R_1+\frac{1}{j\omega C_1}
\end{equation}

\begin{equation}
Z_{out}=\frac{1}{\frac{1}{R_2}+\frac{j\omega}{\frac{1}{C_2}+\frac{1}{C_3}}}
\end{equation}

\noindent
where $\omega$ is the angular frequency of the input signal.\\

In addition, we obtained the following transfer function:

\begin{equation}
T_f(\omega)=\frac{v_{out}}{v_{in}}=\frac{j\omega R_1 C_1}{1+j\omega R_1 C_1}\times \left(1+\frac{R_3+R_5}{R_4}\right) \times \frac{1}{1+j\omega R_2 C_2}
\end{equation}

With it, we were able to determine the frequency response, $v_{out}(f)/v_{in}(f)$. The gain and phase plots in terms of the frequency of the input signal are presented in Figures~\ref{fig:gainteo} and~\ref{fig:phaseteo}.

\noindent
\begin{minipage}[c]{0.5\linewidth}

\begin{figure}[H] \centering
\includegraphics[width=1\linewidth]{gainoctave.eps}
\caption{Frequency Response - Gain.}
\label{fig:gainteo}
\end{figure}

\end{minipage}
\begin{minipage}[c]{0.5\linewidth}

\begin{figure}[H] \centering
\includegraphics[width=1\linewidth]{phaseoctave.eps}
\caption{Frequency Response - Phase.}
\label{fig:phaseteo}
\end{figure}

\end{minipage}

\vspace{4mm}

In order to compute the central frequency, we determined the low and high cutoff frequencies, $f_l$ and $f_h$, respectively. Hence, the central frequency, $f_c$, is given by:

\begin{equation}
f_c=\sqrt{f_l\times f_h}
\end{equation}

Finally, we calculated the voltage gain, as well as the input and output impedances at the central frequency. Considering that the goal was to achieve a central frequency of 1000 Hz and a gain of 40 dB, we also computed the corresponding deviations. The results of the theoretical analysis are presented in Table~\ref{tab:resultsteo}.

\begin{table}[H]
  \centering
  \begin{tabular}{|l|r|}
    \hline    
    \input{../mat/octavetab}
  \end{tabular}
  \caption{Results obtained with Octave.}
  \label{tab:resultsteo}
\end{table}

