\section{Introduction}
\label{sec:introduction}

The objective of this laboratory assignment is to design a BandPass Filter, which increases the input voltage, making it as efficient as possible in what concerns both the cost and functioning. \\

In order to achieve this goal, we made use of an Operational Amplifier (model uA741 OP-AMP). In addition, we had at our disposal three $1k\Omega$, three $10k\Omega$ and three $100k\Omega$ resistors, as well as three $220 nF$ and three $1 \mu F$ capacitors. In the OP-AMP integrated circuit, we also used transistors and diodes. The costs of all of the mentioned components are presented in Table~\ref{tab:costs}.

\begin{center}
\begin{tabular}{ | c | c | }
\hline
\textbf{Component} & \textbf{Cost} \\
\hline
Resistors & 1 MU/k$\Omega$ \\  
Capacitors & 1 MU/$\mu$F \\
Transistors & 0.1 MU/transistor \\
Diodes & 0.1 MU/diode \\
\hline   
\end{tabular}
\captionof{table}[]{Cost of the components of the circuit.}
\label{tab:costs}
\end{center}

The following quantity, merit $M$, is defined with the purpose of evaluating the quality of the BandPass Filter:

\vspace{-2mm}

\begin{equation}
  M = \frac{1}{cost \times (gain deviation \times central frequency deviation + 10^{-6})}
\end{equation}

\vspace{2mm}

Thus, the higher the value of the function, the better its quality.\\

With this in mind, we built the circuit illustrated in Figure~\ref{fig:bandpass}, with the components shown in Table~\ref{tab:components}.

\begin{figure}[H] \centering
\includegraphics[width=1\linewidth]{bandpass_filter.pdf}
\caption{BandPass Filter.}
\label{fig:bandpass}
\end{figure}

\vspace{2mm}

\begin{center} 
\begin{tabular}{ | c | c | }
\hline
\textbf{Component} & \textbf{Value} \\
\hline
$V_{in}$ & $0.01\sin{(2000\pi t)}$ V \\
\hline
$R_1$ & $1$ $k\Omega$ \\
\hline
$R_2$ & $1$ $k\Omega$ \\
\hline
$R_3$ & $100$ $k\Omega$ \\
\hline
$R_4$ & $1$ $k\Omega$ \\
\hline
$R_5$ & $10$ k$\Omega$ \\
\hline
$C_1$ & $220$ $nF$ \\
\hline
$C_2$ & $220$ $nF$ \\
\hline
$C_3$ & $220$ $nF$ \\
\hline
\end{tabular}
\captionof{table}[]{Values of the components of the circuit.}
\label{tab:components}
\end{center}

In Section~\ref{sec:analysis}, we present a theoretical analysis of the BandPass Filter, where we develop a model able to predict the output voltage gain, as well as the input and output impedances at the central frequency. Moreover, we get the theoretical frequency response ($v_{out}(f)/v{in}(f))$. In Section~\ref{sec:simulation}, we write an Ngspice script to simulate the BandPass Filter and to measure the aforementioned quantities. Additionally, we compare the results from Section~\ref{sec:analysis} to the ones obtained by simulation. Furthermore, in Section~\ref{sec:conclusion}, we summarise the results of the simulations and their agreement with the theoretical predictions. Finally, we present the merit $M$ of the designed BandPass Filter.

