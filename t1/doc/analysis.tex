\section{Theoretical Analysis}
\label{sec:analysis}

In this section, the circuit shown in Figure~\ref{fig:t1} is analysed
theoretically, using mesh and nodal methods.

\subsection{Mesh Analysis}

Mesh analysis is a well-organized method for solving a circuit, which makes use of Kirchhoff’s Voltage Law and Ohm's Law to deduce a set of equations guaranteed to be solvable if the circuit has a solution.

In order to solve the circuit in study resorting to this method, we started by arbitrarily assigning mesh currents in the essential meshes, $I_1$ and $I_3$ looping in a clockwise direction and $I_2$ and $I_4$ in an anti-clockwise direction, as shown in the Figure~\ref{fig:mesh}.

\vspace{-8mm}

\begin{figure}[H] \centering
\includegraphics[width=0.6\linewidth]{mesh_method.pdf}
\caption{Mesh Analysis.}
\label{fig:mesh}
\end{figure}

Hence, applying KVL to meshes $1$, $2$ and $3$ and equaling $I_4$ to current source value $I_d$, we deduced the following system of linear equations, whose solution is the set of assigned currents:

\vspace{1mm}

\[
{\begin{bmatrix}
K_b R_3 & K_b R_3 - 1 & 0 & 0\\
0 & 0 & 0 & 1\\
R_1 + R_3 + R_4 & R_3 & -R_4 & 0\\
-R_4 & 0 & R_4 + R_6 + R_7 - K_c & 0\\
            \end{bmatrix}
            }
{\begin{bmatrix}
I_1\\
I_2\\
I_3\\
I_4\\
            \end{bmatrix}
            }
    =
{\begin{bmatrix}
0\\
I_d\\
V_a\\
0\\
            \end{bmatrix}
            }
\]


\vspace{1mm}

The solution vector for this system was obtained using the Octave maths tool and is presented in Table~\ref{tab:meshvec}:

\begin{table}[H]
  \centering
  \begin{tabular}{|l|r|}
    \hline    
    {\bf Name} & {\bf Value [A]} \\ \hline
    \input{../mat/mesh}
  \end{tabular}
  \caption{Solution vector for assigned currents.}
  \label{tab:meshvec}
\end{table}

From these data, we were able to determine the rest of the circuit's characteristics, which are shown in Table~\ref{tab:meshtab}:

\begin{table}[H]
  \centering
  \begin{tabular}{|l|r|}
    \hline    
    {\bf Name} & {\bf Value [A or V]} \\ \hline
    \input{../mat/meshtab}
  \end{tabular}
  \caption{Circuit's characteristics using mesh method. A variable preceded by @ is of type {\em current}
    and expressed in Ampere; other variables are of type {\it voltage} and expressed in
    Volt.}
  \label{tab:meshtab}
\end{table}


\subsection{Node Analysis}

Similarly, node analysis is used to deduce a set of equations to determine the nodes potential in an electric circuit, this time using Kirchhoff’s Current Law and Ohm's Law.

In order to solve the circuit in study resorting to this method, we applied KCL in nodes not connected to voltage sources, $2$, $3$, $6$ and $7$, and wrote additional equations for nodes related by voltage sources, $1$, $4$ and $5$. This way, we were able to write 6 linearly independent equations, for what we needed one more to determine all the seven variables (excluding GND). Therefore, we managed to relate nodes $5$ and GND using KCL to get the last equation needed, and obtained the following system of linear equations, whose solution is the set of nodes potentials:

\[
{\begin{bmatrix}
1 & 0 & 0 & -1 & 0 & 0 & 0\\
0 & 0 & 0 & -\frac{K_c}{R_6} & 1 & 0 & \frac{K_c}{R_6}\\
\frac{1}{R_1} & -\frac{1}{R_1}-\frac{1}{R_2}-\frac{1}{R_3} & \frac{1}{R_2} & 0 & \frac{1}{R_3} & 0 & 0\\
0 & -\frac{1}{R_2}-K_b & \frac{1}{R_2} & 0 & K_b & 0 & 0\\
0 & -K_b & 0 & 0 & \frac{1}{R_5}+K_b & -\frac{1}{R_5} & 0\\
0 & 0 & 0 & \frac{1}{R_6} & 0 & 0 & -\frac{1}{R_6}-\frac{1}{R_7}\\
0 & \frac{1}{R_3} & 0 & \frac{1}{R_4} & -\frac{1}{R_3}-\frac{1}{R_4}-\frac{1}{R_5} & \frac{1}{R_5} & \frac{1}{R_7}\\
            \end{bmatrix}
            }
{\begin{bmatrix}
V_1\\
V_2\\
V_3\\
V_4\\
V_5\\
V_6\\
V_7\\
            \end{bmatrix}
            }
    =
{\begin{bmatrix}
V_a\\
0\\
0\\
0\\
-I_d\\
0\\
I_d\\
            \end{bmatrix}
            }
\]

Since GND potential isn't an unknown variable, taking the value 0, we excluded it from this system, in order to simplify the calculations.

Once again, the solution vector for this system was obtained using Octave. The results are shown in Table ~\ref{tab:nodetab}:

\begin{table}[H]
  \centering
  \begin{tabular}{|l|r|}
    \hline    
    {\bf Name} & {\bf Value [V]} \\ \hline
    \input{../mat/node}
  \end{tabular}
  \caption{Solution vector for nodes' potentials.}
  \label{tab:nodetab}
\end{table}

Like above, we were able to calculate the rest of the circuit's characteristics, which are presented in Table~\ref{tab:nodetab}:

\begin{table}[H]
  \centering
  \begin{tabular}{|l|r|}
    \hline    
    {\bf Name} & {\bf Value [A or V]} \\ \hline
    \input{../mat/nodetab}
  \end{tabular}
  \caption{Circuit's characteristics using nodal method. A variable preceded by @ is of type {\em current}
    and expressed in Ampere; other variables are of type {\it voltage} and expressed in
    Volt.}
  \label{tab:nodetab}
\end{table}
















