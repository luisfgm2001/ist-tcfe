\section{Introduction}
\label{sec:introduction}

% state the learning objective 
The objective of this laboratory assignment is to study a RC circuit containing eight nodes numbered from 1 to 3 and from 5 to 6, plus GND. In the present circuit, we can find seven resistors, $R_1$ to $R_7$, one capacitor, $C$, one voltage source, $v_s$, one voltage controlled current source, $I_b$ and, finally, one current controlled voltage source, $V_d$, all connected as seen in Figure~\ref{fig:t1}. For $t <= 0s$, the voltage source supplies constant voltage, however, for $t > 0s$, it supplies a sinusoidal voltage.

\begin{figure}[H] \centering
\includegraphics[width=0.8\linewidth]{t2.pdf}
\caption{Electric circuit.}
\label{fig:t1}
\end{figure}

The previously known characteristics of the circuit are summarized in Table~\ref{tab:datatab}:

\begin{table}[H]
  \centering
  \begin{tabular}{|l|r|}
    \hline    
    {\bf Name} & {\bf Value} \\ \hline
    \input{../mat/datatab}
  \end{tabular}
  \caption{Previously known characteristics.}
  \label{tab:datatab}
\end{table}

In Section~\ref{sec:analysis}, we present a theoretical analysis of the circuit, where we start by determining the voltages in every node and the currents in every branch for $t<0s$, using the nodal method. In addition, the equivalent resistance, $R_{eq}$, is determined as seen from the capacitor terminals, in order to compute the natural solution $v_{6n}(t)$. Using the mentioned method, we also calculate the phasor voltages in each node, so we are able to compute the forced solution, as well as the final solution by superimposing the two previous ones. Finally, we analyse the variation of the magnitude and phase of the voltages in node 6 and in the capacitor with the frequency of the voltage source. In Section~\ref{sec:simulation}, we compare the results from Section~\ref{sec:analysis} with the ones obtained by simulation using Ngspice. Furthermore, in Section~\ref{sec:conclusion}, we summarise the results of the simulations and their agreement with the theoretical predictions.
