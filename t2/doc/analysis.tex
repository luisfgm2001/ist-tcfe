\section{Theoretical Analysis}
\label{sec:analysis}

\subsection{Node Analysis (t<0s)}
\label{subsec:nodeanalysis}

In order to solve the circuit in study resorting to the nodal method, we applied KCL to nodes not connected to voltage sources, $2$, $3$, $6$ and $7$, and wrote additional equations for nodes related by voltage sources, $1$, $5$ and $8$. This way, we were able to write 6 linearly independent equations, for what we needed one more to determine all the seven variables (excluding GND). Therefore, we managed to apply KCL to the super-node formed by node 1 and GND to get the last equation needed, and obtained the following system of linear equations, whose solution is the set of nodes potentials:

\[
{\begin{bmatrix}
1 & 0 & 0 & 0 & 0 & 0 & 0\\
-\frac{1}{R_1} & \frac{1}{R_1}+\frac{1}{R_2}+\frac{1}{R_3} & -\frac{1}{R_2} & -\frac{1}{R_3} & 0 & 0 & 0\\
0 & \frac{1}{R_2}+K_b & -\frac{1}{R_2} & -K_b & 0 & 0 & 0\\
0 & K_b & 0 & -\frac{1}{R_5}-K_b & \frac{1}{R_5} & 0 & 0\\
0 & 0 & 0 & 0 & 0 & \frac{1}{R_6}+\frac{1}{R_7} & -\frac{1}{R_7}\\
0 & 0 & 0 & 1 & 0 & \frac{K_d}{R_6} & -1\\
-\frac{1}{R_1} & \frac{1}{R_1} & 0 & \frac{1}{R_4} & 0 & \frac{1}{R_6} & 0\\
            \end{bmatrix}
            }
{\begin{bmatrix}
V_1\\
V_2\\
V_3\\
V_5\\
V_6\\
V_7\\
V_8\\
            \end{bmatrix}
            }
    =
{\begin{bmatrix}
V_s\\
0\\
0\\
0\\
0\\
0\\
0\\
            \end{bmatrix}
            }
\]

Since GND potential isn't an unknown variable, taking the value 0, we excluded it from this system, in order to simplify the calculations. Additionally, we considered that the voltage source was turned on for a long time, so the capacitor was fully charged and the current in this component was $0A$.

The solution vector for this system was obtained using the Octave maths tool. The results are shown in Table ~\ref{tab:node1}:

\begin{table}[H]
  \centering
  \begin{tabular}{|l|r|}
    \hline    
    {\bf Name} & {\bf Value [V]} \\ \hline
    \input{../mat/node1}
  \end{tabular}
  \caption{Solution vector for nodes' potentials.}
  \label{tab:node1}
\end{table}

Knowing the voltages in all nodes, we were able to calculate the rest of the circuit's characteristics, which are presented in Table~\ref{tab:nodetab1}:

\begin{table}[H]
  \centering
  \begin{tabular}{|l|r|}
    \hline    
    {\bf Name} & {\bf Value [A or V]} \\ \hline
    \input{../mat/nodetab1}
  \end{tabular}
  \caption{Circuit's characteristics using nodal method. A variable preceded by @ is of type {\em current}
    and expressed in Ampere; other variables are of type {\it voltage} and expressed in
    Volt.}
  \label{tab:nodetab1}
\end{table}

\subsection{Equivalent Resistance, $R_{eq}$}
\label{subsec:eqresist}

In order to determine the equivalent resistance as seen from the capacitor terminals, we analysed a circuit with $vs=0$ and a voltage source $Vx = V(6)-V(8)$ in the capacitor's place, where $V(6)$ and $V(8)$ are the voltages determined in Subsection\ref{subsec:nodeanalysis}. Afterwards, we calculated the current, $I_x$, supplied by $V_x$ and computed the equivalent resistance of the form $R_{eq}=V_x/I_x$.

We needed this procedure, because the circuit is too complex, not being possible to determine directly the equivalent resistance, which was needed to calculate the time constant, $\tau$. Furthermore, this procedure was essential to determine the initial conditions for the natural solution, $v_{6n}(t)$.

As in the previous subsection, to compute the voltages in each node, we used the nodal method, applying KCL to nodes not connected to voltage sources, $2$, $3$ and $7$, and writing three more equations for nodes related by voltage sources. Finally, we applied KCL to the supernode formed by node 1 and GND.

\[
{\begin{bmatrix}
1 & 0 & 0 & 0 & 0 & 0 & 0\\
-\frac{1}{R_1} & \frac{1}{R_1}+\frac{1}{R_2}+\frac{1}{R_3} & -\frac{1}{R_2} & -\frac{1}{R_3} & 0 & 0 & 0\\
0 & \frac{1}{R_2}+K_b & -\frac{1}{R_2} & -K_b & 0 & 0 & 0\\
0 & 0 & 0 & 0 & 0 & \frac{1}{R_6}+\frac{1}{R_7} & -\frac{1}{R_7}\\
0 & 0 & 0 & 1 & 0 & \frac{K_d}{R_6} & -1\\
0 & 0 & 0 & 0 & 1 & 0 & -1\\
-\frac{1}{R_1} & \frac{1}{R_1} & 0 & \frac{1}{R_4} & 0 & \frac{1}{R_6} & 0\\
            \end{bmatrix}
            }
{\begin{bmatrix}
V_1\\
V_2\\
V_3\\
V_5\\
V_6\\
V_7\\
V_8\\
            \end{bmatrix}
            }
    =
{\begin{bmatrix}
0\\
0\\
0\\
0\\
0\\
V_x\\
0\\
            \end{bmatrix}
            }
\]

After we determined the voltages in every node, we calculated the current $I_x$ by applying KCL to node 6:

\begin{equation}
  I_x = K_b(V_2-V5)+\frac{V_6-V_5}{R_5}
\end{equation}

Also, the time constant, $\tau$, is given by:

\begin{equation}
  \tau = R_{eq}C
\end{equation}

Additionally, the voltages determined resorting to the mentioned method, as well as the current, $I_x$, the equivalent resistance, $R_{eq}$, and the time constant, $\tau$, are presented in Table~\ref{node2}.

\begin{table}[H]
  \centering
  \begin{tabular}{|l|r|}
    \hline    
    {\bf Name} & {\bf Value} \\ \hline
    \input{../mat/node2}
  \end{tabular}
  \caption{Nodes' voltages determined using nodal method (expressed in Volt), current $I_x$ (expressed in Ampere), equivalent resistance, $R_{eq}$ (expressed in $\Omega$) and time constant, $\tau$ (expressed in $s$).}
  \label{tab:node2}
\end{table}


\subsection{Natural Solution, $v_{6n}(t)$}

As learned in the theory classes, the natural solution in a RC circuit is of the form:

\begin{equation}
  v_{6n}(t) = v_6(+\infty)+(v_6(0^+)-v_6(+\infty))e^{-\frac{t}{\tau}}
  \label{eq:natsol}
\end{equation}

The value of $v_6(0^+)$ was already determined and is presented in Table~\ref{tab:node2}. Considering that the voltage source, $v_s$ is turned off, it is known that, when $t \rightarrow \infty$, the capacitor will discharge and the voltages in the nodes will go to zero, so $v_6(+\inf)=0$. Replacing in Equation~\ref{eq:natsol}, we get the solution illustrated in Figure~\ref{fig:natsol}.

\begin{figure}[H] \centering
\includegraphics[width=0.8\linewidth]{natural6.eps}
\caption{Natural solution, $v_{6n}(t)$, (expressed in Volt) in the interval [0,20]ms.}
\label{fig:natsol}
\end{figure}


\subsection{Forced Solution, $v_{6f}(t)$} \label{subsec:forsol}

In the forced condition, the voltages in all nodes are sinusoidal signals with the same frequency as the voltage source $v_s$. We can analyse this circuit by considering that each voltage and current is a complex exponential, taking its real part for the real solution. Therefore, the voltage in an arbitrary node i satisfies the equation:

\begin{equation}
  v_i(t) = V_i e^{j(\omega t - \phi_i)}
\end{equation}

where $\omega = 2\pi f$ is the angular frequency and $\phi_i$ the phase delay of the voltage in node i. 

Knowing this, we can write the voltage source as $v_s(t) = e^{j(\omega t - \pi/2)}$.
 
Also, the phasor voltage of each node i is given by:

\begin{equation}
  \widetilde{V}_i = V_i e^{-j\phi_i}
\end{equation}

Finally, since the impedance of the capacitor is $Z_c = \frac{1}{j \omega C}$, we have that the current in the capacitor going from node 6 to node 8 is:

\begin{equation}
  i_c(t) = j \omega C (v_6(t)-v_8(t))
\end{equation}

Applying the nodal method, we can write four equations using KCL in nodes $2$, $3$, $6$ and $7$, which are not connected to voltage sources, two equations for nodes related to voltage sources and one more aplying KCL to the super-node formed by node 1 and GND. The system of equations, whose solution is the vector of the phasor voltages, is the following:

\[
{\begin{bmatrix}
1 & 0 & 0 & 0 & 0 & 0 & 0\\
-\frac{1}{R_1} & \frac{1}{R_1}+\frac{1}{R_2}+\frac{1}{R_3} & -\frac{1}{R_2} & -\frac{1}{R_3} & 0 & 0 & 0\\
0 & \frac{1}{R_2}+K_b & -\frac{1}{R_2} & -K_b & 0 & 0 & 0\\
0 & 0 & 0 & 0 & 0 & \frac{1}{R_6}+\frac{1}{R_7} & -\frac{1}{R_7}\\
0 & 0 & 0 & 1 & 0 & \frac{K_d}{R_6} & -1\\
-\frac{1}{R_1} & \frac{1}{R_1} & 0 & \frac{1}{R_4} & 0 & \frac{1}{R_6} & 0\\
0 & K_b & 0 & -K_b-\frac{1}{R_5} & \frac{1}{R_5}+jC\omega & 0 & -jC\omega
            \end{bmatrix}
            }
{\begin{bmatrix}
\widetilde{V}_1\\
\widetilde{V}_2\\
\widetilde{V}_3\\
\widetilde{V}_5\\
\widetilde{V}_6\\
\widetilde{V}_7\\
\widetilde{V}_8\\
            \end{bmatrix}
            }
    =
{\begin{bmatrix}
\widetilde{V}_s\\
0\\
0\\
0\\
0\\
0\\
0\\
            \end{bmatrix}
            }
\]

Once again, the solution vector for this system was obtained using Octave. The results are shown in Table ~\ref{tab:phasortab}:

\begin{table}[H]
  \centering
  \begin{tabular}{|c|c|c|c|c}
    \hline    
    {\bf Node} & {\bf Phasor, $\widetilde{V}$} & {\bf Norm, $V$} & {\bf Phase, $\phi$} \\ \hline
    \input{../mat/phasortab}
  \end{tabular}
  \caption{Phasor voltages, norm of the phasor (expressed in Volt) and phase of the phasor (expressed in degrees).}
  \label{tab:phasortab}
\end{table}

Taking the real part of the complex exponential determined in Subsection~\ref{subsec:forsol}, we get that the forced solution is:

\begin{equation}
  v_{6f}(t) = V_6 cos(\omega t - \phi_6)
\end{equation}


\subsection{Final Solution, $v_6(t)$} \label{subsec:finsol}

For $t<=0$, the voltage source remains constant and the voltages in all nodes were already determined in Subsection~\ref{subsec:nodeanalysis}. For t>0, the voltage source corresponds to a sinusoidal source and the natural and forced solutions were determined in Subsection~\ref{subsec:natsol}. This being said, the final solution is the superimposition of both solutions and the results are illustrated in Figure~\ref{fig:finsol}.

\begin{figure}[H] \centering
\includegraphics[width=0.8\linewidth]{alinea5.eps}
\caption{Final solution, $v_{6}(t)$, and the voltage in the source, $v_s(t)$, (expressed in Volt) in the interval [-5,20]ms.}
\label{fig:finsol}
\end{figure}


\subsection{Frequency Response}

In the previous subsections, we considered that the voltage source had a frequency $f=1kHz$. Now, we will analyse how the eletric circuit responds for different frequencies, $f$. For each frequency, we performed the procedure described before, where we computed the phasor voltages in every node resorting to nodal method.

In Figure~\ref{fig:magnitude}, we present the variation of the magnitude of $v_6(t)$, $v_s(t)$ and $v_c(t)$ in decibels, for frequency range $0.1Hz$ to $1MHz$.

\begin{figure}[H] \centering
\includegraphics[width=0.8\linewidth]{alinea6_1.eps}
\caption{Magnitude response of $v_6(t)$(red), $v_s(t)$(blue) and $v_c(t)$(green) (expressed in decibels).}
\label{fig:magnitude}
\end{figure}

It turns out that the magnitude of the voltage source is constant and equal to 0, which makes sense, since it is the base signal for converting the voltages to decibels. It is evident that the magnitude in node 6 remains approximately constant for small and high frequencies, although there is an interval in between, where it decreases from a value bigger than the magnitude of the source to one that is smaller. The voltage in the capacitor also remains constant for small frequencies, however, for higher frequencies, it decreases linearly (in decibels). Consequently, this quantity in Volt tends to zero when the frequency increases.

In Figure~\ref{fig:phase}, we present the phase delay of the voltage in node 6 and in the capacitor with respect to the voltage source, for the same frequency range.

\begin{figure}[H] \centering
\includegraphics[width=0.8\linewidth]{alinea6_2.eps}
\caption{Phase response of $v_6(t)$(red), $v_s(t)$(blue) and $v_c(t)$(green) (expressed in degrees).}
\label{fig:phase}
\end{figure}

First of all, we see that the phase of the voltage source remains constant and equal to 0, since the plot presents the phase differences with respect to it. For the voltages of node 8 and the capacitor, the phases remain constant for small and high frequencies, even though there is an interval, just like it happens with the magnitudes, where the phases vary. Additionally, we can conclude that, for small frequencies, the voltages are both in phase with the voltage source, however, for high frequencies, the voltage of node 8 is in opposition phase with the source, as the difference of phases is $-180^{\circ}$, and the phase of the voltage in the capacitor differs $-90^{\circ}$ from the source.

